%%%%%%%%%%%%%%%%%%%%%%%%%%%%%%%%%%%%%%%%%%%%%%%%%%%%
%Portada
%%%%%%%%%%%%%%%%%%%%%%%%%%%%%%%%%%%%%%%%%%%%%%%%%%%%

\begin{titlepage}


\begin{figure}[ht]
\hspace{0.1\linewidth}
\begin{minipage}[b]{0.75\linewidth}
\begin{center}
\textbf{\Large{UNIVERSIDAD DE SANTIAGO DE CHILE}}\\
\textbf{FACULTAD DE CIENCIA}\\
\textbf{DEPARTAMENTO DE MATEMÁTICA Y CIENCIA DE LA COMPUTACIÓN}
\end{center}	
\end{minipage}
\begin{minipage}[b]{0.1\linewidth}
\centering
\includegraphics[width=2cm]{Imágenes/logo5}
\end{minipage}
\end{figure}

\begin{center}
\vspace{2.5cm}

% Título de Trabajo
\begin{center}
\Large{\textbf{Grafos cuasi-aleatorios y el Lema de regularidad de Szemerédi}}
\end{center}
\smallskip

% Nombre Autor
\begin{center}
{\textbf{Felipe Andrés Sánchez Erazo}}
\end{center}
\vspace{2.5cm}

% Profesor Guía
\hspace{0.45\linewidth}
\begin{minipage}[b]{0.45\linewidth}
\begin{flushleft}
\textbf{Profesor(a) Guía:}\\
Hi\d{ê}p Hàn
% si su tesis tiene profesor(a) co-guía, descomente las siguientes líneas
%\\
%\textbf{Profesor(a) Co-Guía:}\\
%escriba el nombre del profesor(a) co-guía.
\end{flushleft}
\end{minipage}

\vspace{2cm}

% Protocolo
\hspace{0.45\linewidth}
\begin{minipage}[b]{0.45\linewidth}
\begin{flushleft}
Trabajo de titulación para optar
al \\
título profesional de Ingeniero Matemático
\end{flushleft}
\end{minipage}

\vspace{1.5cm}

\begin{center}
Santiago - Chile \\
2024
\end{center}
%\setlinespacing{1.5}

\end{center}
\end{titlepage}
\pagenumbering{roman}
%%%%%%%%%%%%%%%%%%%%%%%%%%%%%%%%%%%%%%%%%%%%%%%%%%%%
%Derechos de Autor
%%%%%%%%%%%%%%%%%%%%%%%%%%%%%%%%%%%%%%%%%%%%%%%%%%%%
  © Felipe Andrés Sánchez Erazo. \\
Se  autoriza  la  reproducción  parcial  o  total  de  esta  obra  con  fines 
académicos,  por  cualquier  forma,  medio  o  procedimiento,  siempre  y 
cuando se incluya la cita bibliográfica del documento. 

\newpage


%%%%%%%%%%%%%%%%%%%%%%%%%%%%%%%%%%%%%%%%%%%%%%%%%%%%
%Hoja de Evaluación 
%%%%%%%%%%%%%%%%%%%%%%%%%%%%%%%%%%%%%%%%%%%%%%%%%%%%
%
%

\begin{figure}[ht]
\hspace{0.1\linewidth}
\begin{minipage}[b]{0.75\linewidth}
\begin{center}
\textbf{\Large{UNIVERSIDAD DE SANTIAGO DE CHILE}}\\
\textbf{FACULTAD DE CIENCIA}\\
\textbf{DEPARTAMENTO DE MATEMÁTICA Y CIENCIA DE LA COMPUTACIÓN}
\end{center}	
\end{minipage}
\begin{minipage}[b]{0.1\linewidth}
\centering
\includegraphics[width=2cm]{Imágenes/logo5}
\end{minipage}
\end{figure}

\begin{center}


% Título de Trabajo
\begin{center}
\Large{\textbf{Grafos cuasi-aleatorios y el Lema de regularidad de Szemerédi}}
\end{center}


% Nombre Autor
\begin{center}
{\textbf{Felipe Andrés Sánchez Erazo}}
\end{center}


\smallskip
\begin{center}
Trabajo de Titulación presentado a la Facultad de Ciencia, en cumplimiento parcial de los requisitos exigidos para optar al título de \textbf{Ingeniero Matemático}.\\
Este trabajo de Graduación fue presentado bajo la supervisión del profesor guía Dr. Hi\d{ê}p Hàn, 
%y del (de la) profesor(a) co-guía Dr. XXX
del Departamento de Matemática y Ciencia de la Computación de la Facultad de Ciencia.
\end{center}



\begin{center}
\hspace{0.45\linewidth}
\begin{minipage}[b]{0.9\linewidth}
\begin{flushright}
\begin{tabular}{lcr}
  &&\\
  & \hspace{2cm} & \rule{6cm}{1pt}\\
&\hspace{2cm}  & Dr. Hi\d{ê}p Hàn\\
  &&\textit{Profesor Guía}\\
  &&\\
%& \hspace{2cm} & \rule{6cm}{1pt}\\
%&\hspace{2cm}  & Dr(a). nombre profesor(a) co-guía.\\
% &&\textit{Profesor(a) Co-Guía}\\
%  &&\\
& \hspace{2cm} & \rule{6cm}{1pt}\\
&\hspace{2cm}&  Dr. Marcos Kiwi Krauskopf\\
&\hspace{2cm}&\textit{Profesor Informante}\\
  &&\\
\rule{5cm}{1pt} & \hspace{2cm} & \rule{6cm}{1pt}\\
Pablo Marín Álvarez&\hspace{2cm}&  Dr. Sebastián Barbieri Lemp\\
\textit{Director del Departamento}&&\textit{Profesor Informante}
\end{tabular}
\end{flushright}
\end{minipage}
\end{center}

\end{center}



\newpage



%%%%%%%%%%%%%%%%%%%%%%%%%%%%%%%%%%%%%%%%%%%%%%%%%%%%
% INFORMES DE TESIS
%%%%%%%%%%%%%%%%%%%%%%%%%%%%%%%%%%%%%%%%%%%%%%%%%%%%

%agregue aquí los informes realizados por la comisión
% use el comando includepdf  para tal efecto

%\includepdf[pages=-]{./Informe_tesis/Informe.pdf}
%\includepdf[pages=-]{.pdf}
%\includepdf[pages=-]{.pdf}
%\includepdf[pages=-]{.pdf}
%\includepdf[pages=-]{.pdf}


%\newpage








%%%%%%%%%%%%%%%%%%%%%%%%%%%%%%%%%%%%%%%%%%%%%%%%%%%%
%Dedicatoria
%%%%%%%%%%%%%%%%%%%%%%%%%%%%%%%%%%%%%%%%%%%%%%%%%%%%

\begin{flushright}
\ \vfil\vfil
\textit{A mi abuelo, Sergio Sánchez}
\end{flushright}
\newpage

%%%%%%%%%%%%%%%%%%%%%%%%%%%%%%%%%%%%%%%%%%%%%%%%%%%%
%Agradecimientos
%%%%%%%%%%%%%%%%%%%%%%%%%%%%%%%%%%%%%%%%%%%%%%%%%%%%
%\chapter{Agradecimientos}

%Dōmo arigatōgozaimasu!

\newpage

%%%%%%%%%%%%%%%%%%%%%%%%%%%%%%%%%%%%%%%%%%%%%%%%%%%%
%Tabla de contenidos
%%%%%%%%%%%%%%%%%%%%%%%%%%%%%%%%%%%%%%%%%%%%%%%%%%%%

% Construye los índices automáticamente
%\tableofcontents
%\listoftables
%\listoffigures


%%%%%%%%%%%%%%%%%%%%%%%%%%%%%%%%%%%%%%%%%%%%%%%%%%%%
%Resumen
%%%%%%%%%%%%%%%%%%%%%%%%%%%%%%%%%%%%%%%%%%%%%%%%%%%%
%\chapter{Resumen}


%%%%%%%%%%%%%%%%%%%%%%%%%%%%%%%%%%%%%%%%%%%%%%%%%%%%
%Introducción
%%%%%%%%%%%%%%%%%%%%%%%%%%%%%%%%%%%%%%%%%%%%%%%%%%%%
%\mainmatter% INDICA COMIENZO DE PARTE PRINCIPAL DE LA TESIS
%\chapter*{Introducción}
%\addcontentsline{toc}{chapter}{Introducción}


